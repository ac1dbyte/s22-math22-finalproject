\documentclass[12pt, titlepage]{article}
\usepackage{amssymb}
\usepackage{amsmath, xcolor}
\usepackage{amsfonts}
\usepackage{csquotes}
\usepackage{mathtools}
\usepackage{systeme}
\usepackage{hyperref}
\usepackage{amsthm}
\usepackage{tcolorbox}
\usepackage{blindtext}
\usepackage{framed}
\usepackage{cite}
\DeclarePairedDelimiter{\ceil}{\lceil}{\rceil}
\DeclarePairedDelimiter{\floor}{\lfloor}{\rfloor}
\let\oldproof\proof
\renewcommand{\proof}{\oldproof[\unskip\nopunct]}
\newcommand{\R}{\mathbb{R}}
\newcommand{\Z}{\mathbb{Z}}
\newcommand{\N}{\mathbb{N}}
\newcommand{\Q}{\mathbb{Q}}
\newcommand{\D}{\mathbb{D}}
\newcommand{\B}{\mathbb{B}}
\newcommand{\thm}[2] {\begin{tcolorbox}[colframe=darkgray] {\bf #1:} {\itshape #2} \end{tcolorbox}}
\newcommand{\pf}[1] {\begin{tcolorbox}[colback=white,colframe=darkgray]\proof {\bf Proof:} #1 \end{tcolorbox}}
\title{\bf Code Obfuscation}
\author{Mario Fares and Austin Wilson}
\begin{document}

\maketitle{}

% \begin{center}
    {\huge Cover Letter}
\end{center}
\hspace{\parindent} For our draft review, most of our peer reviewers expressed
an
interest in our
topic, stating that it was interesting and that the introduction sets the paper
up really well. We were happy upon learning this and took several steps to keep
the content approachable and grounded in linear algebra, but we still wanted to
make the topic seem as practical as possible. On that note, our biggest
additions to our paper are Python code samples meant to illustrate some of the
concepts and functions we discuss. After all, the paper is still about code
obfuscation. We opted not to include pictures of the code, but rather include
it as formatted text that is noticeable but integrate well with the rest of the
formatting. In addition, we made several changes throughout the paper based on
the comments we received.

First, for the introduction, we added a specific example, based on Sid Satish's
comment, of a company having high performance code that would give them an edge
over their competitors. This clarifies why making code harder to read is
sometimes crucial. Furthermore, based on our TF Eunice Sakarto's feedback, we
also added a brief explanation of our paper's outline.

Second, for the mixed boolean arithmetic section, we made sure to explain in
more depth the bitwise operators we use and their primary function. In
addition, we added the truth table that was merely mentioned in the draft.
Thank you to Zoe Price, James Strong, Connor Yu, Shiloh Liu, and Eunice Sakarto
for your clear and specific feedback on this section.

Third, for the data types section, we added a very brief code snippet and
explanation of certain data types based on Sid Satish's comment. Otherwise,
the comments on this part were good so we tried to keep the language and the
examples simple. We also decided to add several code snippets of Python
functions that are equivalent to the mathematical functions we defined. We made
sure any code included in the paper is correct by testing it ourselves.

We sincerely hope that our paper inspires both the programmers and
non-programmers of Math 22a to dive into code and create their own
obfuscations. We made it a point for the code presented in the paper to remain
approachable and understandable and so, anyone can copy it with little
modification and get it to work. Thank you for reading and we hope you
enjoy the paper!
\newpage
\begin{center}
    {\huge Cover Letter}
\end{center}
\hspace{\parindent} For our draft review, most of our peer reviewers expressed
an
interest in our
topic, stating that it was interesting and that the introduction sets the paper
up really well. We were happy upon learning this and took several steps to keep
the content approachable and grounded in linear algebra, but we still wanted to
make the topic seem as practical as possible. On that note, our biggest
additions to our paper are Python code samples meant to illustrate some of the
concepts and functions we discuss. After all, the paper is still about code
obfuscation. We opted not to include pictures of the code, but rather include
it as formatted text that is noticeable but integrate well with the rest of the
formatting. In addition, we made several changes throughout the paper based on
the comments we received.

First, for the introduction, we added a specific example, based on Sid Satish's
comment, of a company having high performance code that would give them an edge
over their competitors. This clarifies why making code harder to read is
sometimes crucial. Furthermore, based on our TF Eunice Sakarto's feedback, we
also added a brief explanation of our paper's outline.

Second, for the mixed boolean arithmetic section, we made sure to explain in
more depth the bitwise operators we use and their primary function. In
addition, we added the truth table that was merely mentioned in the draft.
Thank you to Zoe Price, James Strong, Connor Yu, Shiloh Liu, and Eunice Sakarto
for your clear and specific feedback on this section.

Third, for the data types section, we added a very brief code snippet and
explanation of certain data types based on Sid Satish's comment. Otherwise,
the comments on this part were good so we tried to keep the language and the
examples simple. We also decided to add several code snippets of Python
functions that are equivalent to the mathematical functions we defined. We made
sure any code included in the paper is correct by testing it ourselves.

We sincerely hope that our paper inspires both the programmers and
non-programmers of Math 22a to dive into code and create their own
obfuscations. We made it a point for the code presented in the paper to remain
approachable and understandable and so, anyone can copy it with little
modification and get it to work. Thank you for reading and we hope you
enjoy the paper!
\newpage
\section{Introduction}
In essential programming doctrine, it is considered ``best practice'' to write code that
is concise, understandable, and simple. This paradigm is great for sharing ideas
with other programmers, but doesn't hold up when we consider some business and
security realities of the world. For one, having unknown people understand
your code is not always part of the goal; consider a company that produces
high-performance code where much of their business comes from the fact that
other companies cannot reproduce their algorithm. If they sell or
release their software, they would want it to be done in a way that others still
cannot understand or reproduce its innerworkings.
Similarly, if code is easy to understand and contains any vulnerabilities,
those vulnerabilities will be easier to spot and exploit by unknown, possibly
malicious actors. Even for relatively secure code, understandability can reveal
too much of its innerworkings to the public and result in an attack.
\par So, programmers want to write code that is understandable to the people
they work with to enable collaboration, but they want to release software that is not
able to be reverse engineered. How can we synthesize
these conflicting goals? The primary method is {\itshape software obfuscation}.
Obfuscation is a process performed on code that transforms it into other code
which performs essentially the same operations, but is much more difficult to
understand or emulate. There are several technical ways to do this, obfuscating
the control flow of the program or using clever hardware/system-specific tricks
to dissuade reverse engineers. But these won't always help us hide special values
and computations that might also be sensitive information--for that, we'll have to
look to linear algebra.

In this paper, we discuss two techniques of code obfuscation. We first
introduce bitwise operators and mixed boolean arithmetic to obfuscate
expressions. We then discuss introduce data types and explain how they and
their operations can be represented as matrices. In both sections, we provide a
simple function that we attempt to obfuscate using the aforementioned
techniques in order to contrast them. We also provide throughout the paper
Python code samples that are equivalent to our mathematical functions.

\section{Mixed Boolean Arithmetic}
\subsection{Introduction to Mixed Boolean Arithmetic}
In computer programming, numbers are represented
not in decimal but in binary with a limited number of bits allocated to the 
variable. That means each number is represented in memory as a bit-vector of
the form $\{0,1\}^n$, or $\B^n$ where $\B$ is for binary, and where $n$ is the ``word size'', that is, the maximum 
size of a given number. To compute the decimal value of a given bit vector $v \in \B^n$, 
take the following sum with $v_i$ representing the $i$th bit of $v$:
\begin{align*}
    v = \sum_{i = 0}^{n-1} v_i \cdot 2^{i}
\end{align*}
For our purposes, that means the most significant (largest value) bit is the topmost
or first element of $v$ (and we say that $v_0$ is the bottom-most element. Note that
this is different from how vectors would normally be indexed in linear algebra, but we
do this to make it simpler).
This means that, for example, the number $2$ in $\B^4$ would be represented as 
\begin{align*}
    v = \begin{bmatrix}
        0 \\ 0 \\1 \\0
    \end{bmatrix} = \sum_{i=0}^{3}v_i \cdot 2^{i} = 2^0 \cdot 0 + 2^1 \cdot 1 + 2^2 \cdot 0 + 2^3 \cdot 0 = 2
\end{align*}
This is just a way of interpreting the bit vector. In other words, we could also say that 
if the word size is 4, then $2 = 0010$ in binary. All of these expressions are
equivalent ways of representing the same value.
\par Using these bit-vector numbers, we can use different
operations based on how we treat them, I.E. using bitwise operators such as
$\ll$ and $\gg$ for shifting bits right and left, $\land$ and $\lor$ (bitwise AND and OR), $\neg$ (bitwise NOT), 
and $\oplus$ (bitwise XOR), or using traditional arithmetic operators $+, -, \times,$ etc.
\par When using bitwise operators, it is certainly more clear to treat the 
number as a bit-vector, while when using traditional arithmetic operators
we get the same result by taking the result of normal arithmetic over the modular ring 
($\Z/2^n\Z$). This is because when doing arithmetic on bit-vectors of a set size, 
any time the result we get is larger than can be represented in $n$ bits, the result
is taken modulo $2^n$. For the purposes of our paper, we will intermingle using the 
bit vector and decimal representations of a number, as this ultimately gives us the 
same result and allows us to mix bitwise and arithmetic operators in the same
expression.
\par As an aside for those less comfortable with the bitwise operators defined above, 
this is an example of a truth table, which shows the given values of binary 1-bit
numbers on the left, and in the next columns the corresponding value of an 
expression including some of the operators we've defined above.
\begin{align*}
    \begin{tabular}{c|c|c|c|c|c}
        $x$ & $y$ & $\neg x$ & $x \oplus y$ & $x \land y$ & $x \lor y$\\
        \hline
        0 & 0 & 1 & 0 & 0 & 0\\
        1 & 0 & 0 & 1 & 0 & 1\\
        0 & 1 & 1 & 1 & 0 & 1\\
        1 & 1 & 0 & 0 & 1 & 0\\
    \end{tabular}
\end{align*}
\thm{Definition 1} {
    A bitwise expression $e_j$ is defined as an expression constituted by
    bitwise operators $\land, \lor, \oplus,$ or $\neg$ on some set of variables
    $(x_1, \dots, x_t)$. For example, consider some expressions in the vector space $\B^4$ consisting
    of bit vectors with $4$ bits, with the following:
    \begin{align*}
        x &= \begin{bmatrix}
        0 & 1 & 1 & 1
    \end{bmatrix}^T = 7 & y &= \begin{bmatrix}
        1 & 0 & 1 & 0
    \end{bmatrix}^T = 10
    \\
    x \lor y &= \begin{bmatrix}
        1 & 1 & 1 & 1
    \end{bmatrix}^T = 15
             & x \oplus y &= \begin{bmatrix}
                  1 & 1 & 0 & 1
              \end{bmatrix}^T = 13
    \\
        x \land y &= 
        \begin{bmatrix}
            0 & 0 & 1 & 0
        \end{bmatrix}^T = 2 & \neg y &= \begin{bmatrix}
        0 & 1 & 0 & 1
        \end{bmatrix}^T
        = 5
    \end{align*}
}
Note that there are other bitwise operators such as left and right shift, but for our
purposes we will only use the four in this definition. Consider also doing normal arithmetic operations on bit vectors:
\begin{align*}
x + y &= \begin{bmatrix}
0 & 0 & 0 & 1 
\end{bmatrix}^T = 1 = 17 \mod 2^4 
\end{align*}
\par In this example, a bit should be carried to
the 5th position from the right, but it gets truncated as we're working in $\B^4$, resulting
in $x + y$ being smaller then $x$ or $y$. 
This is why we get the same result as taking the addition in
$\Z/(2^n\Z)$ (Over this ring, we would get $7 + 10 = 17$ and take the result mod $2^4$, 
since $n=4$)
\par Now, to formally define a mixed boolean arithmetic expression:
\thm{Definition 2} {Let $n,s,t \in \N$, where $n$ is the number of bits in a bit vector, 
$s$ is the number of bitwise expressions, and $t$ is the number of 
inputs to the expressions.
Let $x_i$ be a variable over $\B^n$ for
$i = 1,...,t$, and $e_j : (\B^n)^t \rightarrow \B^n$ be a bitwise expression for
$j = 0,...,s-1$. Let $a_j\in \B^n$ be constant coefficients.
Then, a linear MBA (mixed boolean arithmetic) expression $e$ is an expression that can be written as follows:
\begin{align*}
    e(x_1, ..., x_t) = \sum_{j=0}^{s-1} a_j e_j(x_1,...,x_t)
\end{align*}
}
\par Since we've now defined the the essential operators we have in mixed-boolean 
arithmetic
expressions, we can imagine several situations where a programmer might use expressions 
that they would rather keep secret. In cryptography-relevant code, we may use any
number of cryptographic methods that are composed of MBA 
expressions
and wish to keep the exact constituent operations hidden 
to dissuade attackers. For example, let's say we are working on a 
program that uses the following function $F : (\B^{32})^3 \rightarrow \B^{32}$ as part
of a top-secret cryptography algorithm:
\begin{align*}
    F(x, y, z) = 22(x - y + \neg z)
\end{align*}
This means  that $F$ takes three bit-vectors of size 32 bits, $x, y, z \in \B^{32}$, 
and returns the bit vector equivalent to that given expression. 
(We can use 2's complement representation for values which we want to be negative,
which means that for some bit vector $v \in \B^n$, $-v = \neg v + 1$,
and a number is negative if and only if its first bit is set to 1.
This allows us to represent negative numbers using only 0's and 1's in a bit vector. 
Also note here that the results of a bitwise operator do {\itshape not change} 
using 2's complement, only the  interpretation of the result. The same goes for 
arithmetic operators the underlying bit vector representation.)
By the definition we've given above, we know that $F$ is an MBA expression. 
In the next section, 
we will take some tools from linear algebra and use them to obfuscate $F$.
\subsection{Mixed Boolean Arithmetic in Linear Algebra}
Now, there are two ways to re-write an MBA expression as something more complex:
re-writing and inserting identities. Inserting identities means to tag onto
the expression some MBA expression that is always 0, and re-writing means to 
replace an expression with one that is equivalent. First, let's look at methods
of generating and proving identities. There is one critically important theorem 
that will allow us to generate arbitrary MBA identities:
\thm{Theorem 1}{Let $e$ be a linear MBA expression with the same variable definitions
as in Definition 2. Then let $(v_{0,j},...,v_{2^t-1, j})^T$ be the column vector
of the truth table of the Boolean expression from $e_j$, and let 
$A = (v_{i,j})_{2^t \times s}$ be the $\B$-matrix of those truth tables over ($\Z/2^n\Z$).
Let $Y_{s \times 1} = (y_0, \dots, y_{s-1})^T$ be a vector of $s$ variables over
$(\Z/2^n\Z)$. Then $e = 0$ if and only if $AY = 0$ has a solution with $Y = (a_0, ..., a_s-1)^T$.}
{\textbf Proof:} Then, we need to prove that $AY = 0$ with $Y = (a_0, ..., a_{s-1})^T \implies e = 0$. 
    Let $A Y = 0$ with $Y = (a_0, ..., a_{s-1})^T$. Let $f_j(x_{1,i}, ..., x_{t, i})$
    represent the bitwise expression for $e_j$ on the $i$th bits of $(x_1, ..., x_t)$, 
    such that $e_j(x_1, ..., x_t) = \begin{bmatrix}
        f_j(x_{1,0}, ..., x_{t, 0})\\
        \vdots \\
        f_j(x_{1,n-1}, ..., x_{t, n-1})
    \end{bmatrix}$
    Now, we can write
    $e$ as:
    \begin{align*}
        \sum_{j=0}^{s-1}a_j \cdot e_j(x_1, ..., x_t) &= \sum_{j=0}^{s-1} a_j \cdot \begin{bmatrix}
        f_j(x_{1,0}, ..., x_{t, 0})\\
        \vdots \\
        f_j(x_{1,n-1}, ..., x_{t, n-1})
        \end{bmatrix}\\
                                                     &= \sum_{j=0}^{s-1} a_j \sum_{i=0}^{n-1} f_j(x_{1,i}, ..., x_{t, i}) \cdot 2^{i}\\
                                                     &= \sum_{i=0}^{n-1} 2^i (\sum_{j=0}^{s-1} a_j \cdot f_j(x_{1,i}, ..., x_{t, i}) )
    \end{align*}
    Then, we know that since $AY = 0$, then
    \begin{align*}
        \sum_{j=0}^{s-1} a_j \cdot f_j(x_{1,i}, ..., x_{t, i}) = 0
    \end{align*}
    for all $i$, since $A$ represents the truth values of all of $f_j$, and thus 
    \begin{align*}
        \sum_{j=0}^{s-1} a_j \sum_{i=0}^{n-1} f_j(x_{1,i}, ..., x_{t, i}) \cdot 2^{i} = 0
    \end{align*}
    and therefore $e = 0$. \qed
\par A lot of difficult notation and background goes into that proof, but 
to highlight the important part, what this theorem says is that constructing
a matrix of truth tables $A$ from some functions $f_j$ on some bit vectors $(x_1, ..., x_t)$, 
and then solving $AY = 0$ for $Y$ allows us to construct an MBA identity.
\par Let's take an example, consider that we wanted to compose an MBA identitity consisting 
of the following expressions:
\begin{align*}
    f_0(x,y) &= x\\
    f_1(x,y) &= y\\
    f_2(x,y) &= x \land \neg y\\
    f_3(x,y) &= x \oplus y\\
    f_4(x,y) &= \neg x\\
    f_5(x,y) &= -1
\end{align*}
Then, we take the truth tables for each of these expressions in $\B$ 
and form them into a column vector, then compose them into a matrix.
In this case, we get the following:
\begin{align*}
    A =
    \begin{bmatrix}
        0 & 0 & 0 & 0 & 1 & 1\\
        0 & 1 & 0 & 1 & 1 & 1\\
        1 & 0 & 1 & 1 & 0 & 1\\
        1 & 1 & 0 & 0 & 0 & 1\\
    \end{bmatrix}
\end{align*}
(Where we've defined -1 to be all 1's as in 2's complement representation)
Then, solving for $Ax = 0$, we can take some solution:
\begin{align*}
    x = \begin{bmatrix}
    -1\\-1\\-2\\1\\-2\\2
    \end{bmatrix}
\end{align*}
So, plugging in our expressions from the functions the matrix was composed of, 
we get the following equation:
\begin{align*}
    - x - y - 2(x \land \neg y) + (x \oplus y) - 2(\neg x) - 2 = 0
\end{align*}
So, we generated an MBA identity by just inputting some expressions we would like to 
have seen in the final result. Great! Now, we did this by composing $A$ under the 
assumption that all of our variables were 1-bit large. If we were working in the $n$-bit
space, $A$ would become much larger and be more difficult to find a solution for. 
Luckily, there's another useful theorem here, which
we will describe the mathematical justification for, but not rigorously prove.
\thm{Theorem 2}{Let $e : (\B^1)^t \rightarrow \B^1$ be an MBA expression such that
    $e = 0$. Then, for any word-size $n$, $e : (\B^n)^t \rightarrow \B^n = 0$.
}
In other words, any MBA identity that holds in $\B$ also holds in $\B^n$.
To briefly look at why this is the case, consider the following in $\B^1$:
\begin{align*}
    e = \sum_{j=0}^{s-1}a_j \cdot e_j(x_1, ..., x_t) &= \sum_{j=0}^{s-1} a_j \cdot \begin{bmatrix}
    f_j(x_{1, 0}, ..., x_{t, 0})
    \end{bmatrix}\\
\end{align*}
Then, we know that for all $(x_{1,0}, ..., x_{t,0}) \in (\B^n)^t$, by our assumption $e = 0$, 
then $f(x_{1,0}, ..., x_{t, 0}) = 0$. Then, extending this to the $n$-bit space:
\begin{align*}
    e = \sum_{j=0}^{s-1}a_j \cdot e_j(x_1, ..., x_t) &= \sum_{j=0}^{s-1} a_j \cdot \begin{bmatrix}
    f_j(x_{1,0}, ..., x_{t, 0})\\
    \vdots \\
    f_j(x_{1,n-1}, ..., x_{t, n-1})
    \end{bmatrix}\\
\end{align*}
Then:
\begin{align*}
    e =  \sum_{j=0}^{s-1} a_j \sum_{i=0}^{n-1} f_j(x_{1,i}, ..., x_{t, i}) \cdot 2^{i}
\end{align*}
By above, we know that $f_j = 0$ when operating on singular bits, which is exactly what
we do here, so:
\begin{align*}
    &= \sum_{j=0}^{s-1} a_j \sum_{i=0}^{n-1} 0 \cdot 2^{i} = 0
\end{align*}
Therefore, an MBA identity that holds in the 1-bit space holds in the $n$-bit space, 
and we can use the results from these identities with any word size! This is 
great for obfuscating our previous function $F$, which works with 32-bit numbers.
\subsection{Obfuscating Mixed Boolean Arithmetic}
Now, using the method in the previous section to generate MBA identities works well,
but we can also use it to find MBA equalities that allow us to rewrite our original
expression. Take the first example:
\begin{align*} 
    &  - x - y - 2(x \land \neg y) + (x \oplus y) - 2(\neg x) - 2 = 0\\
    &\implies - x - y = 2(x \land \neg y) - (x \oplus y) + 2(\neg x) + 2\\
    &\implies x - y = 2(x \land \neg y) - (x \oplus y) + 2(\neg x) + 2 + 2x
\end{align*}
Then, in any MBA expression that contains $(x - y)$, we can replace it with
the expression on the right. Since we can generate arbitrary MBA identities, 
we can compose any matrix that includes some of the operations in whichever
MBA expression we'd like to obfuscate, and then use the resulting equality
to replace expressions with equivalent ones as long as we'd like to! For example, let's
apply this MBA identity to our function $F$ by rewriting the $x-y$ term and calling
the obfuscated function $G$, and look at
the result in python:
\\
\begin{minipage}{.5\textwidth}
    \begin{small}
    \begin{framed}
        \begin{verbatim}
def F(x: int, y: int,
      z: int) -> int:
    return 22*(x - y + ~z)



        \end{verbatim}
    \end{framed}
    \end{small}
\end{minipage}% This must go next to `\end{minipage}`
\begin{minipage}{.5\textwidth}
    \begin{framed}
        \begin{small}
        \begin{verbatim}
def G(x: int, y: int,
      z: int) -> int:
    return 22*(2*(x & ~y) 
            - (x ^ y) 
            + 2*(~x) + 2
            + 2*x + ~z)
        \end{verbatim}
        \end{small}
    \end{framed}
\end{minipage}
The expression on the right is much more difficult to understand to someone 
reading the code after it's published or sold, and we could even repeat the process
several times to obfuscate with the $z$ term and make the expression look as complex 
as we want. To ensure that our results are correct, we tested that the results of 
these expressions are equivalent for several thousand random values and found that
it succeeded each test.
\par It's worthwhile to point out that obfuscating code arbitrarily is not a great idea,
as there are performance concerns once the expression starts getting too big, especially
if we expect to be using it very often. Therefore, it's a matter of cost-benefit
analysis to decide how much to use this kind of MBA obfuscation. 

% OUTLINE:
%   Intro
%       What is Obfuscation?
%       Applications of Obfuscation and Deobfuscation
%   Mixed-Boolean Arithmetic
%       Definition of MBA
%       Representation of MBA as Matrices
%       Obfuscating MBA with linear algebra
%   Data Types
%       Intro to Data Types
%       Representing Data Types as Matrices
%       Obfuscating Data Type Operations with Matrices
\section{Data Types}
\subsection{Introduction to Data Types}
So, we've seen a way to rewrite expressions while using the same kinds of data, 
i.e. bit vectors, but another method of obfuscation is to exchange some type 
of data for another one entirely.g
A data type is a construct in programming that defines what kind of data a
variable can hold and the operations that a programmer can do with this type.
For those who are familiar with programming, integers, floating point numbers,
booleans and strings come to mind. If these names are unfamiliar, consider the
following code example in Python:
\begin{verbatim}
    is_a_math_paper: bool = True
    topic: str = "Obfuscation"
\end{verbatim}
In the above example, we have a variable \textit{is\_a\_math\_paper} that is a
boolean type. It can therefore only hold the Python values \textit{True} and
\textit{False}. The other variable \textit{topic} is of type \textit{str},
which is the data
type for text. Of course, we can represent much more than
these primitive types by creating a new type and defining its basic
operations. We may for example represent a polynomial and define the addition
of two polynomials as adding the coefficients of the corresponding
same degree variable. There are endless possibilities to represent data in
most programming languages.

The discussion of data types is important as creating new types, changing
between them, and using different operations for the same type is one common
method used in code obfuscation. For example, instead of working with a boolean
$ B $ to check a condition, we can use two integers $ p $ and $ q $, both of
which we can manipulate using integer arithmetic. Hence, wherever we use $ B $,
we may now, for example, add the values of $ p $ and $ q $ and if the sum is
even, that would be equivalent to $ B $ being true.

\subsection{Representing Data Types and Their Operations as Matrices}
One method of obfuscation that we can employ is the use of matrices to obscure
data types and their operations. One example would be to represent a
rational number, what would often be a floating point number in programs, as a
matrix.

Let us say we have a variable $ x $, a rational number, that we wish to
represent as a matrix $ A $. We represent this relation as $ x \rightsquigarrow
A $, which means that $ x $ is obfuscated by $ A $. At this point, we need to
define two functions that will help us move between the data type and the
obfuscation.

Our first function is called the abstraction function denoted by $ af: D \to E
$ and is
defined as follows:
\begin{equation*}
    x \rightsquigarrow A \Leftrightarrow x = af (A)
\end{equation*}
Our abstraction function will allow us to retrieve the original value of our
data type using the obfuscation that we have created, in this case $ A $. The
abstraction function is surjective, as there could be several possible
obfuscations that represent the same value. This characteristic allows for less
predictability in the generated code and thus creates less identifiable
patterns that could help someone reverse engineer an application.

Our second function is called the conversion function denoted by $ cf: E \to D
$ and is defined as follows:
\begin{equation*}
    cf (x) = A
\end{equation*}
Our conversion function will take $ x $, what we want to obfuscate, as an input
and output the obfuscation $ A $. This function is the inverse of $ af $.

Now back to the realm of matrices, we need to define our aforementioned
functions. One example would be to have $ cf $ generate an $ A $ such that $ af
$ simply calculates the determinant of $ A $ yielding $ x $. Thus, we may for
example define $ cf $ as follows:
\begin{equation*}
    cf(x)
    =
    \begin{bmatrix}
        x & a \\
        0 & 1
    \end{bmatrix}
\end{equation*}
While the above is our mathematical definition of the $ cf $ function, we can
also define it in Python using the scientific library NumPy, which allows us to
easily represent matrices in code:
\begin{verbatim}
    from random import randint
    from numpy import array

    def cf(x: int) -> np.array:
        # Generate a random integer between 0 and a 1000
        a: int = randint(0, 1000)
        return array([[x, a],
                      [0, 1]])
\end{verbatim}
This function is equivalent to the mathematical definition that precedes it and
is representing a matrix using NumPy's data type \textit{array}, which allows
us to
create an array of arrays, each representing a row in our desired matrix. Note
that in this example, there are no restrictions on what $ a $ can be. In our
Python code specifically, $ a $ will be a random number generated for us that
is between $ 0
$ and $ 1000 $. Now, the abstraction
function would simply be:
\begin{equation*}
    af \left(
    \begin{bmatrix}
        a & b \\
        c & d
    \end{bmatrix}
    \right)
    =
    \det
    \begin{bmatrix}
        a & b \\
        c & d
    \end{bmatrix}
    = ad - bc
\end{equation*}
In Python:
\begin{verbatim}
    def af(A: np.array) -> int:
        x = linalg.det(A) # Calculate the determinant
        return x          # Return the calculated value
\end{verbatim}
We can now test our 2 functions:
\begin{equation*}
    af(cf(x)) = x \cdot 1- a \cdot 0 = x
\end{equation*}
It works! This is not enough, however, as we still need to define operations
that correspond to basic arithmetic. Let us take addition as an
example. We need a function \textbf{plus} such that:
\begin{equation*}
    x + y \rightsquigarrow \mathbf{plus(A, B)} : x \rightsquigarrow A, y
    \rightsquigarrow B
\end{equation*}
\textbf{plus} can then be defined as:
\begin{equation*}
    plus\left(
    \begin{bmatrix}
        x & a \\
        0 & 1
    \end{bmatrix},
    \begin{bmatrix}
        y & b \\
        0 & 1
    \end{bmatrix}
    \right)
    =
    \begin{bmatrix}
        x + y & a + b \\
        0     & 1
    \end{bmatrix}
\end{equation*}
Translating our math to code:
\begin{verbatim}
    def plus(A: np.array, B: np.array) -> np.array:
        x_11: int = A.item(0, 0) + B.item(0, 0)
        x_12: int = A.item(0, 1) + B.item(0, 1)
        return array([[x_11, x_12],
                      [   0,    1]])
\end{verbatim}
In programming in general, we start counting at 0 instead of 1. Hence, when we
write \textit{A.item(0, 0)}, we want the element in the first row and first
column
of
$ A $. \textit{A.item(0, 1)} is thus the element in the first row and second
column. In the spirit of our plus function, we also define a \textbf{times}
function:
\begin{equation*}
   times\left(
   \begin{bmatrix}
       x & a \\
       0 & 1
   \end{bmatrix},
   \begin{bmatrix}
       y & b \\
       0 & 1
   \end{bmatrix}
   \right)
   =
   \begin{bmatrix}
       x \cdot y & a  \cdot b \\
       0     & 1
   \end{bmatrix}
\end{equation*}
The equivalent Python code:
\begin{verbatim}
    def times(A: np.array, B: np.array) -> np.array:
        x_11: int = A.item(0, 0) * B.item(0, 0)
        x_12: int = A.item(0, 1) * B.item(0, 1)
        return array([[x_11, x_12],
                      [   0,    1]])
\end{verbatim}
Now since our main purpose is the exploration of different obfuscation
techniques, recall this equation from section 2:
\begin{align*}
    F(x, y, z) = 22(x - y + \neg z)
\end{align*}
We obfuscated this function using mixed boolean arithmetic, but we can also use
the operations and functions we have defined so far in this section. One more
function is required though, which is the unary negation function ($ \neg $).
Let us first define it:
\begin{equation*}
    unegate\left(
    \begin{bmatrix}
        x & a \\
        0 & 1
    \end{bmatrix}
    \right)
    =
    \begin{bmatrix}
        -x - 1 & -a - 1  \\
        0      & 1
    \end{bmatrix}
\end{equation*}
Now, using our conversion, times, plus, and unegate functions, we get the
following matrices:
\begin{align*}
    \neg z &=
    \begin{bmatrix}
        -z - 1 & 2 \\
        0      & 1
    \end{bmatrix}
    &x &=
    \begin{bmatrix}
        x  &  3 \\
        0  &  1
    \end{bmatrix} \\
    -y & =
    \begin{bmatrix}
        -y & 5\\
        0  & 1
    \end{bmatrix} &
    x + (-y) &=
    \begin{bmatrix}
        x - y & 8 \\
        0      & 1
    \end{bmatrix} \\
     (x + (- y)) + (\neg z) &=
     \begin{bmatrix}
         x - y -z - 1 & 10 \\
         0            & 1
     \end{bmatrix} &
     22 &=
     \begin{bmatrix}
         22 & 20 \\
         0  & 1
     \end{bmatrix} \\
     22((x + (- y)) + (\neg z)) &=
     \begin{bmatrix}
         22 \cdot (x - y -z - 1) & 200 \\
         0 & 1
     \end{bmatrix}
\end{align*}
Finally, bringing all of these expression together, we get a function $ G $,
which is the obfuscated version of our function $ F $:
\begin{equation*}
    G(x, y, z) =
    \begin{bmatrix}
        22 \cdot (x - y -z - 1) & 200 \\
        0 & 1
    \end{bmatrix}
\end{equation*}
Applying the abstraction function $ af $ on this matrix should return a number
that is
the same result as applying $ F $ to the inputs $ x, y, z $.

We can define many more operations beyond addition and multiplication such as
division, exponentiation, etc. It is important for
these operations to be defined for our matrices to make our obfuscated code
harder to reverse engineer. It would not be much use to simply represent
numbers as
matrices and then convert them back whenever we want to perform arithmetic or
other operations. Moreover, the way we define each operation should also be
sufficiently complex. It is worth noting, though, that these operations need
not be defined in the same way normal matrix operations like addition,
multiplication, etc., are defined. The programmer has complete control over
these definitions as long as they can prove that they will yield the desired
output wherever they are used.

It is important to acknowledge that the example above, although not to be taken
lightly, is too simple an
obfuscation to be used in the real world. It does, however, give a very good
sense of how
obfuscation works. A better example, as proposed by Drape et al., would be to
have a conversion function where $ x $ is an eigenvalue of a $ 2 \times 2 $
matrix and the second eigenvalue is fixed. If we wish to make our obfuscation
even stronger, we can make our matrix bigger for example. In this case, we might
also use fake control flow that uses the value that was randomly generated in a 
way that appears to be purposeful but actually isn't, so that it is harder to see
that this part of the matrix is never relevant.


\newpage
%\nocite{*}
\bibliography{ref}
\bibliographystyle{ieeetr}

\end{document}

