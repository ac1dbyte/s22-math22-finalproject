\documentclass[12pt, titlepage]{article}
\usepackage{amssymb}
\usepackage{amsmath, xcolor}
\usepackage{amsfonts}
\usepackage{csquotes}
\usepackage{mathtools}
\usepackage{systeme}
\usepackage{hyperref}
\usepackage{amsthm}
\usepackage{tcolorbox}
\DeclarePairedDelimiter{\ceil}{\lceil}{\rceil}
\DeclarePairedDelimiter{\floor}{\lfloor}{\rfloor}
\let\oldproof\proof
\renewcommand{\proof}{\color{blue}\oldproof[\unskip\nopunct]}
\newcommand{\R}{\mathbb{R}}
\newcommand{\Z}{\mathbb{Z}}
\newcommand{\N}{\mathbb{N}}
\newcommand{\Q}{\mathbb{Q}}
\newcommand{\D}{\mathbb{D}}
\newcommand{\thm}[2] {\begin{tcolorbox}[colframe=darkgray] {\bf #1:} {\itshape #2} \end{tcolorbox}}
\newcommand{\pf}[1] {\begin{tcolorbox}[colback=white,colframe=darkgray]\proof {\bf Proof:} #1 \end{tcolorbox}}
\title{\bf Code Obfuscation and Deobfuscation}
\author{Mario Fares and Austin Wilson}
\begin{document}

\maketitle{}

\tableofcontents
% \begin{center}
    {\huge Cover Letter}
\end{center}
\hspace{\parindent} For our draft review, most of our peer reviewers expressed
an
interest in our
topic, stating that it was interesting and that the introduction sets the paper
up really well. We were happy upon learning this and took several steps to keep
the content approachable and grounded in linear algebra, but we still wanted to
make the topic seem as practical as possible. On that note, our biggest
additions to our paper are Python code samples meant to illustrate some of the
concepts and functions we discuss. After all, the paper is still about code
obfuscation. We opted not to include pictures of the code, but rather include
it as formatted text that is noticeable but integrate well with the rest of the
formatting. In addition, we made several changes throughout the paper based on
the comments we received.

First, for the introduction, we added a specific example, based on Sid Satish's
comment, of a company having high performance code that would give them an edge
over their competitors. This clarifies why making code harder to read is
sometimes crucial. Furthermore, based on our TF Eunice Sakarto's feedback, we
also added a brief explanation of our paper's outline.

Second, for the mixed boolean arithmetic section, we made sure to explain in
more depth the bitwise operators we use and their primary function. In
addition, we added the truth table that was merely mentioned in the draft.
Thank you to Zoe Price, James Strong, Connor Yu, Shiloh Liu, and Eunice Sakarto
for your clear and specific feedback on this section.

Third, for the data types section, we added a very brief code snippet and
explanation of certain data types based on Sid Satish's comment. Otherwise,
the comments on this part were good so we tried to keep the language and the
examples simple. We also decided to add several code snippets of Python
functions that are equivalent to the mathematical functions we defined. We made
sure any code included in the paper is correct by testing it ourselves.

We sincerely hope that our paper inspires both the programmers and
non-programmers of Math 22a to dive into code and create their own
obfuscations. We made it a point for the code presented in the paper to remain
approachable and understandable and so, anyone can copy it with little
modification and get it to work. Thank you for reading and we hope you
enjoy the paper!
\newpage
\section{Introduction}
In essential programming doctrine, it is considered ``best practice'' to write code that
is concise, understandable, and simple. This paradigm is great for sharing ideas
with other programmers, but doesn't hold up when we consider some business and
security realities of the world. For one, having unknown people understand
your code is not always part of the goal; consider a company that produces
high-performance code where much of their business comes from the fact that
other companies cannot reproduce their algorithm. If they sell or
release their software, they would want it to be done in a way that others still
cannot understand or reproduce its innerworkings.
Similarly, if code is easy to understand and contains any vulnerabilities,
those vulnerabilities will be easier to spot and exploit by unknown, possibly
malicious actors. Even for relatively secure code, understandability can reveal
too much of its innerworkings to the public and result in an attack.
\par So, programmers want to write code that is understandable to the people
they work with to enable collaboration, but they want to release software that is not
able to be reverse engineered. How can we synthesize
these conflicting goals? The primary method is {\itshape software obfuscation}.
Obfuscation is a process performed on code that transforms it into other code
which performs essentially the same operations, but is much more difficult to
understand or emulate. There are several technical ways to do this, obfuscating
the control flow of the program or using clever hardware/system-specific tricks
to dissuade reverse engineers. But these won't always help us hide special values
and computations that might also be sensitive information--for that, we'll have to
look to linear algebra.

In this paper, we discuss two techniques of code obfuscation. We first
introduce bitwise operators and mixed boolean arithmetic to obfuscate
expressions. We then discuss introduce data types and explain how they and
their operations can be represented as matrices. In both sections, we provide a
simple function that we attempt to obfuscate using the aforementioned
techniques in order to contrast them. We also provide throughout the paper
Python code samples that are equivalent to our mathematical functions.

% OUTLINE:
%   Intro
%       What is Obfuscation?
%       Applications of Obfuscation and Deobfuscation
%   Mixed-Boolean Arithmetic
%       Definition of MBA
%       Representation of MBA as Matrices
%       Obfuscating MBA with linear algebra
%   Data Types
%       Intro to Data Types
%       Representing Data Types as Matrices
%       Obfuscating Data Type Operations with Matrices

\end{document}

