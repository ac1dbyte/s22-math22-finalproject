\begin{center}
    {\huge Cover Letter}
\end{center}
\hspace{\parindent} In this paper, Austin worked on the introduction and the
mixed boolean arithmetic section while Mario worked on the conclusion and data
types section. We both contributed equally to the cover letter and the
references. In addition, we each reviewed the other person's assigned sections
and gave feedback accordingly.


For our draft review, most of our peer reviewers expressed
an
interest in our
topic, stating that it was interesting and that the introduction sets the paper
up really well. We were happy upon learning this and took several steps to keep
the content approachable and grounded in linear algebra, but we still wanted to
make the topic seem as practical as possible. On that note, our biggest
additions to our paper are Python code samples meant to illustrate some of the
concepts and functions we discuss. After all, the paper is still about code
obfuscation. We opted not to include pictures of the code, but rather include
it as formatted text that is noticeable but integrate well with the rest of the
formatting. In addition, we made several changes throughout the paper based on
the comments we received.

First, for the introduction, we added a specific example, based on Sid Satish's
comment, of a company having high performance code that would give them an edge
over their competitors. This clarifies why making code harder to read is
sometimes crucial. Furthermore, based on our TF Eunice Sakarto's feedback, we
also added a brief explanation of our paper's outline.

Second, for the mixed boolean arithmetic section, we made sure to explain in
more depth the bitwise operators we use and their primary function. In
addition, we added the truth table that was merely mentioned in the draft.
Thank you to Zoe Price, James Strong, Connor Yu, Shiloh Liu, and Eunice Sakarto
for your clear and specific feedback on this section.

Third, for the data types section, we added a very brief code snippet and
explanation of certain data types based on Sid Satish's comment. Otherwise,
the comments on this part were good so we tried to keep the language and the
examples simple. We also decided to add several code snippets of Python
functions that are equivalent to the mathematical functions we defined. We made
sure any code included in the paper is correct by testing it ourselves.

We sincerely hope that our paper inspires both the programmers and
non-programmers of Math 22a to dive into code and create their own
obfuscations. We made it a point for the code presented in the paper to remain
approachable and understandable and so, anyone can copy it with little
modification and get it to work. Thank you for reading and we hope you
enjoy the paper!
\newpage