\section{Conclusion}
In this paper, we discussed the importance of and need for code
obfuscation in real world programs. Furthermore, we presented two primary
methods of obfuscation: mixed boolean arithmetic and data type obfuscation
using matrices.

Using mixed boolean arithmetic, we can take some important expressions in our code
which operate on numbers that are represented by binary, and generate new expressions
that produce the same results while being much more opaque to potential reverse 
engineers. We also showed that you can do this first in the 1-bit space, and then
that the same identities hold in the $n$-bit space, so we can generate these
obfuscating expressions very efficiently.

As for data types, we saw that we can obfuscate them using matrices by defining
a conversion and abstraction function to switch between the actual type and its
matrix representation. Furthermore, we defined basic arithmetic operations like
addition and multiplication for these matrices that correspond to the
operations done on the original values. While we primarily discussed
obfuscating rational numbers, it is not impossible that with a little bit of
creativity, we can do the same for boolean and string data types.

There are several other techniques for obfuscation that are relatively simpler
such as renaming our variables, inserting irrelevant code, and adding redundant
operands among others as discussed by Collberg et al. It is important, however,
that the programmer consider the computational \textit{cost} of these
obfuscations. Adding two numbers is very quick, but adding two matrices
depending on the way we defined that operation could be relatively slower.
Obfuscation is an art and one has to strike a balance between making code
harder to read and making code performant enough to be competitive with other
software. Utilizing obfuscation requires substantial thought about the potential
adversaries a programmer is working against and a cost-benefit analysis regarding
factors like time spent obfuscating and the performance cost of doing so.
