\section{Introduction}
In essential programming doctrine, it is considered ``best practice'' to write code that
is concise, understandable, and simple. This paradigm is great for sharing ideas
with other programmers, but doesn't hold up when we consider some business and 
security realities of the world. For one, having unknown people understand
your code is not always part of the goal; consider a company that produces 
high-performance code where much of their business comes from the fact that 
other companies cannot reproduce their algorithm. If they sell or
release their software, they would want it to be done in a way that others still
cannot understand or reproduce its innerworkings.
Similarly, if code is easy to understand and contains any vulnerabilities, 
those vulnerabilities will be easier to spot and exploit by unknown, possibly
malicious actors. Even for relatively secure code, understandability can reveal
too much of its innerworkings to the public and result in an attack.
\par So, programmers want to write code that is understandable to the people 
they work with to enable collaboration, but they want to release software that is not
able to be reverse engineered. How can we synthesize 
these conflicting goals? The primary method is {\itshape software obfuscation}.
Obfuscation is a process performed on code that transforms it into other code
which performs essentially the same operations, but is much more difficult to 
understand or emulate. There are several technical ways to do this, obfuscating
the control flow of the program or using clever hardware/system-specific tricks
to dissuade reverse engineers. But these won't always help us hide special values
and computations that might also be sensitive information--for that, we'll have to 
look to linear algebra.
